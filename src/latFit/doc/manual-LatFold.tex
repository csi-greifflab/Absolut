\documentclass{article}

\usepackage[latin1]{inputenc}
\usepackage[english]{babel}
\usepackage{algorithm}
\usepackage{algpseudocode}  % pseudocode (aus algorithmixs)
\usepackage{graphicx}
\usepackage{amsmath}


\newcommand{\latfold}{\textsc{LatFold}}

\newsavebox{\savepar}
\newenvironment{boxit}{\begin{lrbox}{\savepar}\begin{minipage}[t]{0.8\textwidth}}
                      {\end{minipage}\end{lrbox}\fbox{\usebox{\savepar}}}

\title{\latfold{} - Manual}

\date{}

\author{Martin Mann - University Freiburg \\\\
{http://www.bioinf.uni-freiburg.de/}}

\begin{document}

\maketitle

\input{version-info}

\section{Description}

\latfold{} implements a Monte-Carlo simulation utilizing a Metropolis criterion
(see \cite{Mann:latpack:HFSP08}). The implementation is based on the Energy
Landscape Library \cite{Mann:ELL:BIRD07}. It supports:

\begin{enumerate}
  \item various lattices (see Sec.~\ref{sec:lat})
  \item arbitrary energy functions (see Sec.~\ref{sec:energy})
  \item different move sets (see Sec.~\ref{sec:lat-params})
  \item \ldots
\end{enumerate}

\section{Method}
\label{sec:method}


\subsection{Global Folding Simulation}
\label{sec:MC}

\subsubsection*{Given:}

\begin{tabular}{rcl}
	$S=S_1,\ldots,S_n$ &:& monomer sequence from alphabet $A$ to fold \\
	$E(S,P)$ &:& energy function (see Sec.~\ref{sec:energy})\\
	$N(L)$ &:& set of neighboring structures of $L$ in the energy landscape \\
	$t_{max}$	&:& a maximal simulation time
\end{tabular}

\subsubsection*{Result:}

\begin{tabular}{rcl}
	$L=L_1,\ldots,L_n$	&:& 3D coordinates of the final \\
						&& placement of $S$ in the lattice
\end{tabular}

\subsubsection*{Method:}


\vspace{0.5em}
\begin{algorithm}[H]
\caption{\latfold{} core algorithm}
\begin{algorithmic}[1]
\tiny
	\State $L=L_1,\ldots,L_n$ \Comment{the currently adopted structure}
	\Statex \Comment{initialized with the open chain $L_i = (i,0,0)$}
	\While{simulation end $t_{max}$ not reached}
		\State Select random neighbor $N_r \in N(L)$
		\State $r \in [0,1]$ \Comment{get random number in interval $[0,1]$}
		\If{($r \leq e^{-\frac{E(S,N_r)-E(S,L)}{kT}}$)}
			\State $L \gets N_r$ \Comment{go to neighboring structure}
		\Else \Comment{keep current structure for this step}
		\EndIf
	\EndWhile
	\State report final structure $L$
\end{algorithmic}
\end{algorithm}





\section{Available Lattices}
\label{sec:lat}

Several lattice models can be used to fold a structure. 

\vspace{0.5em}
The currently supported lattice models and the corresponding neighboring
vectors are:

\vspace{0.5em}
\begin{tabular}{c|l|l|c}
	ID & Name & Neighborhood vectors & \#\\
	\hline
	SQR & Square & $\{\pm(1,0,0),\pm(0,1,0)\}$ & 4\\
	CUB & Cubic & $\{\pm(1,0,0),\pm(0,1,0),\pm(0,0,1)\}$ & 6\\
	FCC & Face Centered Cubic & 
	$\left\{\pm(1,1,0),\pm(1,0,1),\pm(0,1,1), \atop
	\pm(1,-1,0),\pm(1,0,-1),\pm(0,1,-1)\right\}$ & 12 
\end{tabular}



\section{Energy Functions}
\label{sec:energy}

\latfold{} supports arbitrary energy functions that are based either on contacts
or on distance intervals. The specification of an energy function has to be
given in text format and defines the allowed sequence alphabet as well.

In general, the energy of a sequence~$S$ of length~$n$ with structure
coordinates~$P$ is determined by
\begin{equation}
	E(S,P) = \sum_{1\leq i+1<j\leq n} e(S_i, S_j, P_i, P_j).
\end{equation}
Here, $e(S_i, S_j, P_i, P_j)$ is a placeholder for the specific evaluation
function that is given for the different types in the following.

\subsection{Contact Based Energy Function}
\label{sec:energy:contact}

A contact based energy function for an alphabet $A$ is defined by an energy
table $E^c : |A|\times|A| \rightarrow \mathcal{R}$ such that 

\begin{equation}
	e_c(S_i, S_j, P_i, P_j) = 
	\left\{
	\begin{array}{ll}
    	E^c[S_i,S_j] & \mbox{ if $P_i$ and $P_j$ are neighbored} \\
    	0 & \mbox{ else }
    \end{array} \right.
\end{equation}

For example, a function like this was used by Lau and Dill to define the widely
used HP-model~\cite{Lau_Dill:89a}.

\subsubsection*{\underline{ Text File Encoding }}

The \latfold{} text file enconding of a contact based energy function consists of
two parts: the alphabet elements and the energy table. A consecutive string of
the alphabet elements in the first line determines the allowed protein sequence
characters (the alphabet) and the dimensions of the energy table that is read
from the remaining file.

An example energy file for the HPNX-model is:

\begin{center}
\begin{boxit}
\small
\begin{verbatim}
HPNX
-4.0  0.0  0.0  0.0
 0.0 +1.0 -1.0  0.0
 0.0 -1.0 +1.0  0.0
 0.0  0.0  0.0  0.0
\end{verbatim}
\end{boxit}
\end{center}


\subsection{Distance Interval Based Energy Function}
\label{sec:energy:distance}

A distance insterval based energy function for an alphabet $A$ is defined by a
consecutive set of~$k$ distance intervals with the upper bounds~$d^{up}_{1\ldots
k}$ and an energy table $E^i_{1\ldots k} : |A|\times|A| \rightarrow \mathcal{R}$
for each of them. Given the distance to interval index function $idx$ we define
the evaluation function

\begin{eqnarray}
	e_i(S_i, S_j, P_i, P_j) & = & E^i_{idx(P_i,P_j)}[S_i,S_j] \\
	idx(P_i,P_j) & = & \arg\min_{k}\;(|P_i-P_j| \leq d^{up}_k)
\end{eqnarray}


\subsubsection*{\underline{ Text File Encoding }}

The \latfold{} text file enconding of a distance interval based energy function
consists of three parts: the alphabet elements, the upper bounds of the intervals
and the energy tables for the interval. A consecutive string of the alphabet
elements in the first line determines the allowed protein sequence characters
(the alphabet) and the dimensions of the energy tables. The second line
contains a whitespace separated list of the upper interval bounds. Their number
sets the number of energy tables read from the remaining file. The interval
bounds are expected to be given in {\AA}ngstroems. For a correct scaling of
the bounds it is necessary to give the average distance of two consecutive
$C_\alpha$-atoms in the underlying model to \latfold{} (see input parameters in
Sec.~\ref{sec:parameter}).

An example energy file that encodes the HPNX-model using a distance interval
based energy function is:

\begin{center}
\begin{boxit}
\small
\begin{verbatim}
HPNX
3.7 3.9 999999

 0.0  0.0  0.0  0.0
 0.0  0.0  0.0  0.0
 0.0  0.0  0.0  0.0
 0.0  0.0  0.0  0.0

-4.0  0.0  0.0  0.0
 0.0 +1.0 -1.0  0.0
 0.0 -1.0 +1.0  0.0
 0.0  0.0  0.0  0.0

 0.0  0.0  0.0  0.0
 0.0  0.0  0.0  0.0
 0.0  0.0  0.0  0.0
 0.0  0.0  0.0  0.0
\end{verbatim}
\end{boxit}
\end{center}

The number $999999$ is used as a placeholder for $+\infty$, i.e. the upper bound
of the last distance interval. It is important to know that the average
$C_\alpha$-distance in the underlying model was 3.8~\AA. Therefore, the resulting
interval energy function corresponds to the contact based energy function of the
previous section; only distances close to the average $C_\alpha$-distance are
taken into account.


\section{Program Parameters}
\label{sec:parameter}

\subsubsection*{\underline{ Input }}

\begin{description}
	\item[-seq] The sequence to fold globally. It has to be conform
	to the alphabet given by the energy file (see {\bfseries -energyFile}).
	\item[-abs] Optional: The absolute move string of the structure to start 
	simulation with.
	\item[-energyFile] A file that encodes the used alphabet and the
	specific energy function (see Sec.~\ref{sec:energy} for format details).
	\item[-energyForDist] If present, the input of {\bfseries -energyFile} will be
	interpreted as distance interval energy function. Otherwise a contact based
	energy functino is expected.
	\item[-energyCalphaDist] Specifies the average distance of two
	consecutive $C_\alpha$-atoms in the underlying model. This value is needed to
	scale the intervals of a distance interval based energy function onto the
	$C_\alpha$-distances of the lattice model in use.
\end{description}

\subsubsection*{\underline{ Simulation Settings }} 

\begin{description}
	\item[-kT] the relative temperature to calculate the Boltzmann weight for a
	given structure within the formula $\exp(-E/kT)$. Thus, the parameter $kT$ 
	sets the product of temperature ($T$) and the energy function specific scaling
	constant ($k$).
	\item[-maxSteps] simulation ends after this number of simulation steps is done
 	\item[-minE] simulation ends if energy gets below or equal the given value
 	\item[-seed] seed for random number generator (uses a system
        independent linear congruent generator)
 	\item[-runs] number of independent folding simulations to perform
\end{description}

\subsubsection*{\underline{ Lattice Settings }} \label{sec:lat-params}

\begin{description}
	\item[-lat] The lattice model to use for the sequential folding. The
	available list of lattice identifiers is given in Sec.~\ref{sec:lat}.
	\item[-moveSet] Sets the move set to use for the present lattice:
	\begin{description}
    	\item[PullM] Pull-moves definied by Lesh et al. \cite{Lesh:RECOMB2003}
    	\item[PivotM] Pivot-moves defined by Madras and Sokal
    	\cite{Madras:Sokal:JSP88}
    \end{description} 
\end{description}



\subsubsection*{\underline{ Output }}

\begin{description}
  \item[-out] The output mode along the folding simulation:
  \begin{description}
    \item[N] no additional output is done (default)
    \item[E] the energy of the structure of each simulation step is printed
    \item[S] structure and energy of each simulation step is given
  \end{description}
  \item[-outFile] Specifies where to write the output of simulations to. 
  		If equal to 'STDOUT' it is written to standard output, otherwise the 
  		given string is assumed to be the filename to write to.
  \item[-outTiming] If present, the used cpu-time is printed.
\end{description}


\subsubsection*{\underline{ Miscellaneous }}

\begin{description}
	\item[-v] Give verbose output during computation.
	\item[-vv] Give extra verbose output during computation.
	\item[-help] Prints the available program parameters.
\end{description}


\section{Contact}

\begin{tabular}{lcr}
	Martin Mann  && \bfseries http://www.bioinf.uni-freiburg.de/\\
	Bioinformatics Group\\
	University Freiburg, Germany \\
\end{tabular}





\begin{thebibliography}{99}
\bibitem{Lau_Dill:89a}{
	  Kit Fun Lau and Ken A. Dill:
	  {\bfseries A Lattice Statistical Mechanics Model of the Conformational 
	  and Sequence Spaces of Proteins}, 
	  \emph{Macromolecules} 1989,
	  {\bfseries 22}(10):3986--3997
	 }
\bibitem{Lesh:RECOMB2003} {
	Lesh, N., Mitzenmacher, M., and Whitesides, S.:
	{\bfseries A complete and effective move set for simplified protein folding},
	In \emph{Proceedings of the seventh annual international conference
	  on Research in computational molecular biology (RECOMB'03)} 2003,
	 188--195.
	}
\bibitem{Madras:Sokal:JSP88} {
	Madras, N. and Sokal, A.~D.:
	{\bfseries The pivot algorithm: A highly efficient {Monte Carlo} method for the
	  self-avoiding walk},
	\emph{Journal of Statistical Physics} 1988,
	{\bfseries 50}, 109--186.
	}
\bibitem{Mann:latpack:HFSP08} {
	Mann, M., Maticzka, D., Saunders, R., and Backofen, R.:
	{\bfseries Classifying protein-like sequences in arbitrary lattice protein models using {LatPack}},
	\emph{HFSP Journal} 2008,
	{\bfseries 2}(6), 396.
	Special issue on protein folding: experimental and theoretical approaches
	}
\bibitem{Mann:ELL:BIRD07} {
	Mann, M., Will, S., and Backofen, R.:
	{\bfseries The Energy Landscape Library - A Platform for Generic Algorithms},
	In \emph{Proceedings of the 1st international Conference on Bioinformatics 
	Research and Development (BIRD'07)} 2007,
	OCG {\bfseries 217}, 83-86.
	}


% \bibitem{Miao.et.al:JMB:H-collapse:04}{
% 	  J.~Miao, J.~Klein-Seetharaman and H.~Meirovitch:
% 	  {\bfseries The Optimal Fraction of Hydrophobic Residues Required to Ensure
% 	  Protein Collapse}, \emph{Journal of Molecular Biology}
% 	  2003,
% 	  {\bfseries 344}:797-811
% 	 }
% \bibitem{Godzik.et.al:JCC:globLat:93}{
% 	  A.~Godzik, A.~Kolinski and J.~Skolnick:
% 	  {\bfseries Lattice Representations of Globular Proteins: How Good Are They?},
% 	  \emph{Journal of Computational Chemistry} 1993,
% 	  {\bfseries 14(10)}:1194-1202
% 	 }
\end{thebibliography}
 

\end{document}